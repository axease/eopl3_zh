\documentclass{book}
\usepackage{nowidow}
\usepackage[asymmetric, papersize={8in,9in}, right=1.016in, hmarginratio=11:5, heightrounded]{geometry}

\usepackage{fontspec}
\setmainfont{QTPalatine}[BoldFont=QTPalatine-Bold, ItalicFont=QTPalatine-Italic]
\setsansfont{Computer Modern Sans}
\setmonofont{Courier}[
BoldFont=QTPalatine-Bold,
ItalicFont=Courier
]

\usepackage{xeCJK}
\setCJKmainfont[BoldFont={WenQuanYi Micro Hei}, ItalicFont={Adobe Kaiti Std}]{Source Han Serif CN}
\setCJKsansfont{Source Han Serif CN}
\setCJKmonofont{Adobe Kaiti Std}
\xeCJKDeclarePunctStyle{mine}{
  enabled-hanging=true,
}
\xeCJKsetup{CJKmath=true, PlainEquation=true, PunctStyle=mine}
\normalspacedchars{/\\-"}

\usepackage{emptypage}

% use package proof
\usepackage{proof}

% also indent at first paragraph
\usepackage{indentfirst}
\setlength{\parindent}{2em}

% customize title format
\usepackage{titlesec}
\titleformat
{\chapter} % command
[hang]
{\bfseries\Large} % format
{\Larger{\Larger{\Larger\thechapter}}} % label
{0.5em} % sep
{
} % before-code
[
] % after-code

% customize header format
\usepackage{fancyhdr}
\renewcommand{\headrulewidth}{0pt}
\renewcommand{\footrulewidth}{0pt}
\renewcommand{\chaptername}{}
\pagestyle{fancy}
\fancyhf{}
\renewcommand\chaptermark[1]{ \markboth{\textit{\thechapter \hskip1em #1}}{} }
\renewcommand\sectionmark[1]{ \markright{\textit{\thesection \hskip1em  #1}} }
\fancyhead[LE,RO]{\thepage}
\fancyhead[RE]{\leftmark}
\fancyhead[LO]{\rightmark}

\usepackage{xpatch}
\makeatletter
\AtBeginDocument{\xpatchcmd{\@thm}{\thm@headpunct{.}}{\thm@headpunct{}}{}{}}
\makeatother

\usepackage{caption}
\usepackage[labelformat=simple]{subcaption}
\renewcommand\thesubfigure{(\alph{subfigure})}
\captionsetup[figure]{labelfont=bf, labelsep=quad, name=图}

\usepackage[perpage]{footmisc}

\usepackage{makeidx}

\usepackage[hyperindex=false]{hyperref}
\hypersetup{
  pdfauthor={Daniel P. Friedman, Mitchell Wand},
  pdftitle={Essentials Of Programming Languages},
  pdfsubject={programming languages},
  pdfkeywords={programming languages},
  pdfpagelabels=ture,
  colorlinks=true,
  urlcolor=blue,
  linkcolor=blue,
  citecolor=red
}

\makeindex
% change separator between page numbers and delimiter between an entry and the
% first page number, the expected behavior is:
% 1. between an entry and a page number, a comma is inserted
% 2. between an entry and a "see" or "seealso", a dot is inserted
% 3. between a page number and a "see" or "seealso", a dot is inserted
% 4. a "see" or "seealso" is always put at the end of an entry page list
\begin{filecontents}{\jobname.mst}
delim_0 " "
delim_1 " "
delim_2 " "
delim_n " "
\end{filecontents}
% general decorator for index page number, taking 3 arguments:
% parameter 1 is prefix
% parameter 2 is suffix
% parameter 3 is page number of index
%
% WARNING: to generate correct index, option "-r" must be specified for
% makeindex
\usepackage{ifthen}
\usepackage{xifthen}
\newcommand*\idxdecorator[3]{#1\unskip, \hyperpage{#3}\ifthenelse{\isempty{#2}}{#2}{~(#2)}}
% redefine see and see also
\renewcommand*{\seename}{\small{参见}}
\renewcommand*{\alsoname}{\small{另见}}

% see(also) command for level 0
\renewcommand*{\see}[2]{\unskip. \emph {\seename } #1}
\renewcommand*{\seealso}[2]{\unskip. \emph {\alsoname } #1}

% see(also) command for level 1
\newcommand*{\seeSublevel}[2]{ (\emph {\seename } #1)}
\newcommand*{\seealsoSublevel}[2]{ ( \emph {\alsoname } #1)}

% see(also) command for level 2
\newcommand*{\seeSubSublevel}[2]{ (\emph {\seename } #1)}
\newcommand*{\seealsoSubSublevel}[2]{ ( \emph {\alsoname } #1)}

% redefine theindex environment to control the style of index chapter in upper
% layer
\makeatletter
\renewenvironment{theindex}
{\parindent\z@
  \parskip\z@ \@plus .3\p@\relax
  \columnseprule \z@
  \columnsep 35\p@
  \let\item\@idxitem}
{}
\renewcommand\@idxitem{\par\hangindent 4ex}
\renewcommand\subitem{\@idxitem \hspace*{10\p@}}
\renewcommand\subsubitem{\@idxitem \hspace*{20\p@}}
\makeatother

% make phantomsection robust
\MakeRobust\phantomsection

\usepackage{enumitem}
\setlist{labelindent=0pt,leftmargin=*}

\usepackage{amssymb}
